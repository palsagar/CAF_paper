A flow configuration that combines the complexities of high density-ratios with the interaction between capillary, viscous and inertial stresses is that of a water droplet falling in air under the influence of gravitational acceleration. The problem is characterized by a combination of Reynolds, Weber and Bond numbres, the definitions of which are as follows : 

\begin{align}
We=\frac{\rho_{\rm air} U^2 d}{\sigma} \quad,\quad Re= \frac{\rho_{\rm air} U d}{\mu_{\rm air}} \quad,\quad Bo=\frac{\left(\rho_{\rm wat}-\rho_{\rm air}\right) g d^2 }{\sigma}
\end{align}

In our particular numerical setup, $We \simeq 3.2 $, $Re \simeq 1455 $ and $Bo \simeq 1.0 $, thus corresponding to that of a $3mm$ diameter raindrop (a relatively large one) falling in air at an approximate terminal velocity of  $8 m/s$ (interpolated from empirical data, refer to  \cite{gunn1949terminal}). The parameters in the problem setup is given in Table \ref{raindropprop}, and the schematic diagram given by Fig. \ref{setup}. The droplet is initially placed at the  
\begin{table}
\begin{center}
\begin{tabular}{ccccccc}
\hline\hline
$\rho_{\rm air}$ & $\rho_{\rm water}$ & $\mu_{\rm air}$ 
& $\mu_{\rm water}$ & $\sigma$ & $d$ & $g$\\
$\left(kg/m^3\right)$ & $\left(kg/m^3\right)$ & $\left(Pa \, s\right)$ 
& $\left(Pa \,s \right)$ & $\left(N/m\right)$ & $(m)$ & $(m /s^{2})$ \\
\hline
1.2 & $0.9982\, 10^3$ & $1.98\,10^{-5}$ & 
$8.9 \, 10^{-4}$ & $0.0728$ & $3\, 10^{-3}$ & $9.81$\\
\hline\hline
\end{tabular}
\caption{Parameter values used in the simulation of a falling water droplet in air. \label{raindropprop}}
\end{center}
\end{table}
% -----
For such a Weber number the capillary forces dominate and the droplet should 
remain approximately spherical. Instead at moderate resolution ($D/h=15$ to 
60 grid points per diameter were tested) and without the momentum-conserving 
scheme described in this paper the droplet deforms catastrophically
as shown on Fig. \ref{cata}. 

% -----
\begin{figure}
\begin{center}
\includegraphics[width=0.35\textwidth]{Figures/setup.pdf}
\end{center}
\caption{Problem setup for falling rain drop test case. Boundary conditions 
at the top and bottom are a uniform inflow and outflow velocity $U_0(t)$. 
Boundary conditions on the side are free slip (no shear stress).}
\label{setup}
\end{figure}
% -----

% -----
\begin{figure}
\begin{center}
\includegraphics[width=0.75\textwidth]{Figures/cata.png}
\end{center}
\caption{Rain drop test case: catastrophic breakup with non-conserving 
formulation, $D/h=30$.}
\label{cata}
\end{figure}
% -----

This effect was already found by \cite{Xiao:2014vs} in a similar case, 
the sudden interaction of a droplet at rest with uniform gas flow. 
For a Weber number also around 3, the droplet should remain near spherical 
but ref. \cite{Xiao:2014vs} reports a similar catastrophic deformation and 
gives the following explanation. To start with, we neglect gravity and viscous 
effects at this relatively large Reynolds number. Also, we are interested 
in steady-state flow. On the axis and near the hyperbolic stagnation point 
at the front of the droplet one has $u_2=0$ for the transverse (radial) 
velocity and for the axial momentum balance
\be
u_1 \partial_1 u_1 = - \frac 1 \rho \partial_1 p.
\nd
Because of the low viscosity and large density ratio, it is not possible 
for the air flow to immediately entrain the water, so the fluid velocity is 
significantly smaller in the water. 
In the air the acceleration near the stagnation point is $U^2/D$ and 
the pressure gradient is
\be
\partial_1 p \sim \rho_{a} U^2/D.
\nd
The pressure gradient in the water is much smaller, however if in a mixed cell 
the water density multiplies the air acceleration $U^2/D$, so that
\be
\partial_1 p \sim \rho_{w} U^2/D,
\nd
then a large pressure gradient results in the mixed cell or cells. This large 
pressure gradient results in a large pressure inside the droplet near the front 
stagnation point, as shown in Figure \ref{FengXiao}. 
This large pressure is balanced by surface tension only for a sufficiently large 
curvature near the droplet front. This explains the presence of a ``dimple'' 
often seen in low resolution simulations of the falling drop. 
% -----
\begin{figure}
\begin{center}
\begin{tabular}{cc}
\includegraphics[width=0.45\textwidth]{Figures/Sagar/16ppd_MC_vc_NON-MC.png} &
\includegraphics[width=0.45\textwidth]
{Figures/Sagar/non_MC_16ppd_pressure_corrected.png}\\
(a) & (b)
\end{tabular}
\end{center}
\caption{The origin of the pressure peak in the front of the droplet. 
(a) The profile of the pressure on the axis a few timesteps after initialisation 
with the standard, non-momentum conserving method (red) and the present method 
(blue). (b) The pressure distribution immediately after the start of the simulation 
using the standard, non-momentum-conserving method. The pressure peak does
not result yet in the formation of a dimple. In all figures  $D/h = 16$.}
\label{FengXiao}
\end{figure}
% -----

% -----
\begin{figure}
\begin{center}
\begin{tabular}{cc}
\includegraphics[width=0.45\textwidth]{Figures/Sagar/pressure_rd_MC.png} &
\includegraphics[width=0.45\textwidth]
{Figures/Sagar/16ppd_pressure_corrected.png}\\
(a) & (b)
\end{tabular}
\end{center}
\caption{ (a) The pressure profile on the axis a few timesteps after 
initialisation with the present 
method at various resolutions: $D/h = 8$ (red) , $16$ (blue) and $32$ (green). 
(b) The pressure distribution immediately after the start of the simulation 
using the present method and   $D/h = 16$.}
\label{FengXiao_corrected}
\end{figure}
% -----

Applying the method described in this paper brings a considerable improvement,
as shown by a comparison of Figures \ref{FengXiao} and \ref{FengXiao_corrected}. 
Simulations fall in three categories: 
those that blow up anyway, those that have a marked peak in kinetic energy as 
a function of time, associated with deformed interface shapes, 
and those that keep physical values of the kinetic energy and smooth shapes.
The most stable combinations of schemes are CIAM advection with superbee limiter
and WY advection with QUICK limiter. 
The CIAM advection with the superbee limiter appears also to be very diffusive. 

% Slightly less stable methods result when one takes $\hat x = x_i$. 
% In that case we observe at low resolution ($D/h=15$) the energy spike shown 
% in Figure \ref{lowres}. The energy spike is 
% associated with a moving bump on the droplet. Using a higher resolution 
% of $D/h=30$ makes the energy spike disappear. 
% Switching to the shifting of the interpolation point $\hat x$ described in 
% Section \ref{tunedinter}, even more stable behavior is observed, down 
% to resolutions of $D/h=8$. 
% -----
% \begin{figure}
% \begin{center}
% \begin{tabular}{cc}
% \includegraphics[width=0.5\textwidth]{Figures/KE.png}
% & \includegraphics[width=0.4\textwidth]{Figures/bump.png} \\
% (a) & (b)
% \end{tabular}
% \end{center}
% \caption{Effect of a slightly unstable setup. (a) The kinetic energy 
% as a function of time exhibits several
% spikes (b) A snapshot of the simulation at the instant of the formation 
% of the first spike. A pointed
% bump forms on the droplet and starts rotating rapidly.}
% \label{lowres}
% \end{figure}
% -----
\begin{figure}
\begin{center}
\begin{tabular}{cc}
\includegraphics[width=0.42\textwidth]{Figures/vel.png}
& \includegraphics[width=0.48\textwidth]{Figures/vort.png} \\
(a) & (b)
\end{tabular}
\end{center}
\caption{Flow field around the 3 mm droplet with 60 grid points per diameter. 
(a) The velocity magnitude. It is seen that even at this highest resolution 
there are only three points in the boundary layer. (b) The vorticity magnitude. 
The marked separation of the boundary layers is observed with 
a more complex vortical region in the wake.}
\label{magn}
\end{figure}
% -----
\newcommand\DDD{{\cal D}}
Visualisation of the flow around the droplets (Figure \ref{magn}) 
contributes to explain why it is so challenging. 
It can be seen that the boundary layers are very thin, questioning our 
approximation that fluid velocity is continuous across the interface. 

In order to validate our simulation, we study the convergence upon grid 
refinement. Three grids are used, with $D/h=15,\,30$ and $60$. A $12$-mm 
cubic box is used with the $3$-mm droplet at the center.  
We study the convergence of the terminal velocity and that of the shape. 
For the latter, we use as a descriptor of the shape the three moments of 
inertia $I_m$ defined by
\be
I_m = \int_\DDD H x_m^2 {\rm d}\X \;, \quad  1 \le m \le 3,
\nd
where $\DDD$ is the domain used for the computation and $x_m$ is relative to
the center of mass. The convergence of the moments of inertia and terminal 
velocity is shown on Figure \ref{converge}. The velocity seems to converge 
to a value around $7\, m/s$, to be compared with a value of $8.06 \,m/s$ 
found by the authors of ref. \cite{gunn1949terminal}. 
This is not surprising since the experimental terminal velocity would be 
modified for a droplet constrained in a finite-size box.
% -----
\begin{figure}
\begin{center}
\begin{tabular}{cc}
\includegraphics[width=0.45\textwidth]{Figures/veloconv.png}
& \includegraphics[width=0.45\textwidth]{Figures/shapeconv.png} \\
(a) & (b)
\end{tabular}
\end{center}
\caption{Convergence of simulations. (a) Evolution of the terminal velocity 
with grid refinement. (b) Evolution of the three moments of inertia with 
grid refinement.}
\label{converge}
\end{figure}
% -----
% -----
\begin{figure}
\begin{center}
\includegraphics[width=0.99\textwidth]{Figures/flatten.png}
%\includegraphics[width=0.99\textwidth]{Figures/Sagar/fig14-001.jpeg}
\end{center}
\caption{Flattening of the droplet with increasing equivalent diameter 
(see text). From left to right $D_e=3, \,4.6,\, 6.4$ and $8\, mm$.}
\label{flatten}
\end{figure}
% -----
From the values of the moments of inertia, the horizontal and vertical 
extents of the droplet (respectively $D_r$ and $D_x$) can be found 
and compared to the values found by the authors of \cite{Reyssat:2007ko}. 
We find $D_r=3.1 \,mm$ and $D_x=2.6 \,mm$, while ref. \cite{Reyssat:2007ko} 
concludes that drops are quasi-spherical for an equivalent diameter 
$D_e \le l_c$ and deformed for $D_e > l_c$, where $l_c =(\sigma/\rho g)^{1/2}$ 
is the capillary length, $l_c = 2.7 \,mm$ for water. 
Indeed repeating the simulations for larger drops we find increased 
flattening as shown on Figure \ref{flatten}. 
%
%\clearpage

