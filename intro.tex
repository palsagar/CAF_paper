Multiphase flows abound in nature, but their stable and accurate computation remains elusive in many cases.
As a case in point, many numerical methods used for two-phase incompressible flow are
strongly unstable for large density contrasts and large Reynolds numbers. 
Experience with such simulations shows that the presence of
surface tension is an aggravating factor. The large density contrasts that are of interest are 
 air/water or gas/liquid-metal, with $\rho_l/\rho_g$ of the order of $10^3$ or $10^4$.
The large density contrasts are a difficulty whether one deals with any of the three major
interface advection methods, Level-Set, Volume-of-Fluid (VOF) or Front-Tracking, or with combinations 
such as CLSVOF.  (The term density contrast is perferable to density ratio since it encompasses ratios both much larger than one and much smaller than one.)

Several methods have been used to alleviate the high-density-contrast difficulties.
It has been observed by several authors that making the momentum-advection method conservative improves the situation. 
%The rationale for this is that momentum conservation is a ``physical'' property of the methods. 
For incompressible flow, momentum-conserving methods have been initially proposed by \cite{rudman98},
and by several other authors since \cite{bussmann2002modeling,desjardins10,raessi12,le13,Vaudor:2017ip}.
These methods have been shown to
improve the stability of the numerical results in various situations. In particular, liquid-gas flows
 with very contrasted densities, as for example in the the process of atomization,
cause serious problems that are resolved by using momentum-conserving methods. 
In that case another oft-suggested solution is to increase the number of 
equations from the standard four equations to five, six or seven equations, 
by introducing new field variables in each phase. 
The addition of one more density $\rho_i$, momentum $\rho_i \U_{i}$ or energy variable
$\rho_i e_i$  increases  the number of equations. 
The authors of ref. \cite{Saurel99} used seven equations, 
those of \cite{allaire02} used five equations and six equations were used in \cite{Pelanti14}. 
The last three references also use a momentum-conserving formulation. 
Several authors, including some of those cited above, have argued that the difficulty may come from 
gas velocities of order $u_g$ being mixed with liquid densities of order $\rho_l$. 
Both the $\rho_l$ and $u_g$ scales are large and the appearance of an unphysical $\rho_l u_g^2$ dynamic pressure scale could create numerical pressure fluctuations of the same order and unphysical pressure spikes, 
as the one nicely illustrated in \cite{xiao2012large} in the front part of a
suddenly accelerated small droplet. One way of avoiding this unphysical mixture of liquid and gas
quantities is to extrapolate liquid and gas pressure and velocity in the ``other'' phase, 
as in the ghost fluid method. This extrapolation strategy was used successfully in \cite{Xiao:2014vs}. 

It may be argued that a way to avoid this numerical diffusion of liquid and gas quantities is to advect 
the volume fraction and the conserved quantities that depend on it (density, momentum and energy) in a consistent manner.
In incompressible flow in which we specialise in this paper, it means that the volume fraction and momentum
or velocity must be advected in the same way. This is equivalent to request 
that the discontinuity of the Heaviside function $H$, marking the phase transition,
should be advected at the same speed as the discontinuity in momentum. 
This can be expressed by the following consistency requirement: 
if momentum is initially exactly proportional to volume fraction, 
it should remain so after advection. 
We call such a method VOF-consistent.

To satisfy this requirement the idea is to solve
the advection equation for momentum with the same numerical scheme that
is used for the VOF color function.
{This consistency property minimizes the unphysical transfer of momentum from one phase to another due to the differences in the numerical schemes used. The consistency is especially important
when dealing with fluids with a large density contrast where a small numerical momentum transfer from the dense phase to the light phase results in large numerical errors in the velocity field which in turn creates numerical instabilities.}

In this work, we present a modification of the classical 
momentum-preserving scheme proposed by \cite{rudman98}
for the case of a staggered grid and VOF method. We also focus more on 
the consistency requirement between VOF and momentum advection, and only 
partially conserve momentum in the scheme. This will illustrate mostly the benefit
of using a consistent advection scheme. 

The paper is organized as follow: the second section deals with the continuum mechanics formulation 
for incompressible flow and sharp interfaces. Section 3 describes our numerical method, starting
with an overview of already-known methods for spatial discretization, time-stepping, and VOF advection. We 
continue with the new momentum advection-VOF-consistent method. Section 4 is devoted to tests of the 
method, followed by a conclusion. 

Among the authors, Gretar Tryggvason, Ruben Scardovelli, Yue Stanley Ling and Stephane
Zaleski have been involved in the construction of the base of the
ParisSimulator VOF and Front-Tracking code that was used to implement
and test the ideas in this paper. ParisSimulator is itself based on a
Front-Tracking code developed by Gretar Tryggvason, Sadegh Dabiri and
Jiacai Lu.  Daniel Fuster was involved in the development of the
Momentum Conserving method with consistent VOF advection, with help
from Tomas Arrufat, Leon Malan and Yue Stanley Ling.  The falling
raindrop testing and the corresponding figures were done by Tomas
Arrufat for the CIAM method and Sagar Pal for the WY
method. The atomisation testing was done by Marco
Crialesi-Esposito. 
